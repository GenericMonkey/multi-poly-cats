\documentclass{article}
\usepackage{mathpartir}
\usepackage{tikz-cd}
\usepackage{enumitem}
\usepackage{wrapfig}
\usepackage{fancyvrb}
\usepackage{comment}

\usepackage{ stmaryrd }

\begin{document}

\title{Lessons from Evil Category Theory in Cubical Agda}

\begin{abstract}
  We describe lessons learned from extending the cubical-agda category
  theory library to support common constructions used in denotational
  semantics of programming languages.
\end{abstract}

\section{Solving Equations using Free and Presented Categories and the Yoneda Embedding}

Performing category theoretic calculations by hand can quickly become extremely tedious. TODO: motivation etc

Free categories, presented categories, quivers vs graphs etc

\section{Universal Properties using Presheaves and Profunctors}

Many category theory textbooks say that every notion of universal
property (terminal objects, adjoint functors, Kan extensions,
representable functors) can represent the other and therefore they are
all equivalent. While this is true, in a proof assistant we would like
to pick one as the basis and then implement the others on top of it
and derive as many useful theorems from our core notion.  In the
cubical library, there were originally several notions developed on an
ad hoc basis: two functors being adjoint, terminal objects, binary
products, right Kan extensions and limits. For our approach we decided
to implement \emph{representable functors} as the main notion of
universal property, as it subsumes other notions very directly without
the need for many intermediate constructions.

To be precise, the universal properties in the library come in the
following topological ordering (after defining categories, functors, natural transformations):
\begin{enumerate}
\item Terminal objects
\item Category of Sets, Functor categories, Yoneda's lemma, Category of elements.
\item Representable functors as universal elements of a presheaf,
  equivalently a terminal object of the category of elements.
\item Bifunctors, funny cartesian product of categories
\item Profunctors as bifunctors into the category of sets. Equivalence
  of representability of profunctors as functors with parameterized
  representability.
\end{enumerate}

These first three definitions follow the usual structure found in
several textbooks, notably, Riehl and ??. The final two are not
usually covered in introductory textbooks but we argue give the best
foundation for other notions, e.g., adjoints and Kan extensions.

First the classical category theory account of profunctors and representability: TODO...

When implementing this in a proof assistant, however, definitional
equalities are of prime importance and getting the right formulation
of bifunctor is surprisingly tricky. As an example consider the
definition of a cartesian product of objects and the exponential. TODO: examples of bad definitions...

\section{Example: Monads}

Why extension systems work better, just like in Haskell...

\end{document}
